Overall structure of the table will be done with 3 primary layers. A database will function as the back end of the whole project and contain the career data. Camera layer will take in user input which is the markers or pucks in this case and the interface layer will display the appropriate data for the user.  

\begin{figure}[h!]
	\centering
 	\includegraphics[width=0.60\textwidth]{images/layers}
 \caption{A simple architectural layer diagram}
\end{figure}

\subsection{Camera Layer Description}
Camera layer will primarily be what takes in user input. For our table the input will be special unique markers placed onto the surface of the table. A camera mounted below the surface of the table will operate in infrared to ignore the light of the projector and read the marker. This marker will be processed and sent into other layers.

\subsection{Interface Layer Description}
The interface layer will mainly be the display and results of the user input. The user will place a unique marker onto the surface of the table in which will be sent to this layer for display. This layer will display the data associated with the marker and allow user interaction for selecting what information they want to view regarding the career. 

\subsection{Database Layer Description}
Database layer will contain the information regarding the careers. It will allow a user to easily create a new career marker, remove a marker, or update information associated with a marker.  
